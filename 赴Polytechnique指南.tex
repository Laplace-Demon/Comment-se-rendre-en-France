%% LyX 2.3.7 created this file.  For more info, see http://www.lyx.org/.
%% Do not edit unless you really know what you are doing.
\documentclass[english]{ctexart}
\usepackage{fontspec}
\usepackage{color}
\usepackage[unicode=true]
 {hyperref}

\makeatletter

%%%%%%%%%%%%%%%%%%%%%%%%%%%%%% LyX specific LaTeX commands.
%% Special footnote code from the package 'stblftnt.sty'
%% Author: Robin Fairbairns -- Last revised Dec 13 1996
\let\SF@@footnote\footnote
\def\footnote{\ifx\protect\@typeset@protect
    \expandafter\SF@@footnote
  \else
    \expandafter\SF@gobble@opt
  \fi
}
\expandafter\def\csname SF@gobble@opt \endcsname{\@ifnextchar[%]
  \SF@gobble@twobracket
  \@gobble
}
\edef\SF@gobble@opt{\noexpand\protect
  \expandafter\noexpand\csname SF@gobble@opt \endcsname}
\def\SF@gobble@twobracket[#1]#2{}
%% Because html converters don't know tabularnewline
\providecommand{\tabularnewline}{\\}

\makeatother

\usepackage{polyglossia}
\setdefaultlanguage[variant=american]{english}
\begin{document}
\title{赴 Polytechnique 指南}

\maketitle
作者鸣谢:X14 - 19 级的同学们,是你们的首当其冲、纷纷掉坑,让作者能够为学弟学妹们分享更多的人生经验,并能够指导一代又一代的孩子们涌向巴黎西南某高校。

\pagebreak{}

\tableofcontents{}

\pagebreak{}

\section{待办手续概述}

以往年的情况来看,综合理工的开学时间是来年四月份,但国际生会被要求在二月中旬到校,以接受一段时间的“浸透式法语教育”。在抵达法国前的一段时间里,学生需要完成以下一些手续:
\begin{description}
\item [{办理护照}] 护照办理是应当被首先提上日程的:一方面,它仅需要身份证和在读证明;另一方面,预约法语预签证考试的网站需要使用护照号注册,而就我个人的经历而言,大量的时间都是在等待考试中度过的。\footnote{TCF考试需要提前一个月预约,只在规定的几个城市举办考试,这其中包含了上海。}
\item [{办理出生公证以及完成双认证}] 出生公证作为签证的材料之一,需要花费一定的时间才能办理。对出生公证进行双认证则需要更多的时间,好在它只在抵达后办理住房补贴时需要用到。
\item [{完成预签证手续}] 长期居留学生签证的申请者需要参加法国高等教育署的预签证程序。预签证是签证的前一步,我认为它相当于是法方对留学目的的认可。在预签证需要的材料中,获取法语语言考试成绩和本科阶段成绩单的认证报告需要较长的时间。
\item [{办理签证}] 在 Etude en France 上办理预签证的同时,就可以前往 France Visa 网站填写签证申请表格,继而使用该网站上的
FRA 号注册 TLScontact 账号,提前预约递交签证的时间。然而在真正前往签证受理中心时\footnote{同样地,只有包含上海在内的某些城市设立了签证受理中心,在预约时就应当确定地点。},前述的步骤都应当完成,它们将作为签证申请材料的一部分被提交。
\item [{完成出国体检、补种疫苗}] 出国体检的目的有三个:办理《国际旅行健康检查证明书》\footnote{俗称“小红本”。}、办理《疫苗接种或预防措施国际证书》\footnote{俗称“小黄本”。}、让医生填写
X 提供的医学表格。它们不是签证材料的一部分,主要是在综合理工注册过程中、以及来法后会用到。如果有一些必需的疫苗尚未接种,也可以趁这个时间接种。
\item [{完成交大的因公出国手续}] 负责这一部分的老师会在关键的时间节点予以提醒、并协助办理因公出国手续,因此并不需要个人过多操心。作为参考,因公出国手续需要一个团长在系统里发起申请、每个人提供家长同意函、以及提交盖齐了章的离校单。
\item [{完成综合理工的注册手续}] 负责这一部分的老师会在关键的时间节点予以邮件提醒。在签证成功后,大家就有条件完成综合理工的注册手续。这一步需要护照扫描件、签证号码、出生公证扫描件、医学表格扫描件、以及
CVEC 的支付证明\footnote{Contribution de vie étudiante et de campus,可以点击\href{https://www.etudiant.gouv.fr/fr/cvec-une-demarche-de-rentree-incontournable-955}{这里}了解。}。
\end{description}

\section{现在可以开始做的事情}

\subsection{交大手续(上)}
\begin{description}
\item [{本科生在读证明}] 在新行政楼 B 楼一楼注册与学务中心免费开具中英文双语版本,请复印若干并扫描为电子版。如果需要加盖校印,请在开具时就告知工作人员,同时要额外等待一天\footnote{加盖校印并不是必需的,详情参见\ref{subsec:=0062A4=007167}一章。}。办理护照时,在读证明的有效期为一个月。
\item [{本科期间成绩单}] 在新行政楼 B 楼一楼注册与学务中心分别开具中文成绩单和英文成绩单,价格是五元一份,请复印若干并扫描为电子版。自
2021 年开始,也可以在\href{https://my.sjtu.edu.cn/}{数字交大}或是交我办上直接申请电子成绩单。请仔细核对自己的拼音名字是否与护照一致。
\end{description}
备注:如果去 X 的时间早于下学期开学时间,就不需要大三上学期的成绩单。法语专业需同时开具法语一专和工科辅修的成绩单。
%
\begin{description}
\item [{上海交大的录取通知书}] 请复印若干并扫描为电子版。\footnote{2020 级的同学发现并没有地方需要使用录取通知书。}
\item [{高考成绩单}] 请复印若干并扫描为电子版。\footnote{醉了吧,还需要这个东西。}
\end{description}
备注:某些省份在高考后可能没有发放纸质高考成绩单,这种情况需要本人或委托人携带身份证与委托书去高考地省招生办开具一份\footnote{事实上,各地政策不同,需要自行询问当地招办。例如 2022 年江苏省就只提供电子证明,并且在网站上随便点一点就能拿到。},且要注意时限。法语保送生无高考成绩单的,可以到交大档案局开保送证明替代高考成绩单。自
2021 年开始可\href{http://dazx.sjtu.edu.cn/}{线上办理保送证明}:在线办理
- 学籍档案查询。
%

\subsection{本科成绩认证\protect\footnote{据杨学长的经验,成绩认证办理用时较长,需要早一点开始。}}

这一项是 Campus France \footnote{下文可能会简称为 CF。}的要求,建议先完成
CF 帐号注册之后,检查过材料清单再决定需不需要办这一项认证。登陆\href{https://www.chsi.com.cn/}{中国高等教育学生信息网(学信网)},点击出国教育背景信息服务
- 网上申请 - 点击进入网上申请系统 - 申请认证,选择中文和英文的大学成绩验证。

认证需要输入所有课程的名称以及成绩信息,便捷起见,可以在\href{http://i.sjtu.edu.cn}{教学信息服务网}
- 信息查询 - 学生成绩查询中点击“导出”,并进行微调以符合格式。在学信网上传的内容需要与成绩单的格式保持基本一致,即“分学期”、“按照课程名称顺序”、“无学时和课程类别”的成绩单。学信网声称以法签为目的的成绩认证需要提供“档案号”,这指的是\ref{subsec:=009884=007B7E=008BC1}中的
CN 号,然而事实证明,不填写档案号也没有关系。

请定期检查手机短信和学信网,以便于及时确认信息。中文成绩单的认证会比英文成绩单的认证早约一周完成,并分别发送通知\footnote{2020 级的同学在系统中质问为何没有英文认证时,惨遭工作人员的阴阳:“建议可等待英文版出具”。}。

\textcolor{red}{注意:对于法语专业的学生,一专成绩单需认证,辅修成绩单无需认证。在申请预签证时,请上传认证后的一专成绩单及未认证的辅修成绩单。成绩单需包含最近学期的成绩。} 

\subsection{Polytechnique 相关手续}

根据经验,在来法之前有一些与 X 相关的事情需要做:
\begin{description}
\item [{填写文件}] Coralie Talet(她是 X 主管外国学生事务的人员之一)会发一堆文件给你们, 有的需要你们先打印出来填写好(
例如: Fiche d’acceuil, Fiche de renseignements, Fiche de ressortissant
étranger, Transferts des données 等等),务必仔细阅读和填写,不清楚的一概可以回邮件问 Coralie。
\item [{获取录取通知书和住宿证明}] X 会统一发给 SPEIT,收到后记得核对一下姓名、入学年份与日期、出生地等信息是否有误。
\item [{选择体育项目}] X 体育组的人会给你发邮件,要求你们选择体育项目,这个大家量力而为仔细看看介绍以后选择就可以了。
\item [{奖助学金手续}] 这个会有 X 财务部门专门人员通过邮件联系你们。
\item [{体检与疫苗}] Coralie 会发送一份体检单与疫苗检查单给你们,需要你们在国内完成,详见\ref{subsec:=004F53=0068C0=004E0E=0075AB=0082D7=0063A5=0079CD}。
\item [{填写网上资料}] 这是 X 的注册手续。只有得到签证后才能够真正完成此网上表格。在填写表格与护照签证相关的部分中,请注意我们是首先拿到的签证,而没有
resident permit,其实在前面填写时问是否有 resident permit 选择“否”后, 自动变为填写 visa 信息,在
document 一栏原本需要上传 resident permit 的扫描件的项目自动也就变成了 visa 扫描件。
\end{description}

\subsection{护照\label{subsec:=0062A4=007167}}

以在上海市办理护照为例。首先访问\href{https://zwdt.sh.gov.cn/govPortals/index}{上海一网通办网站}\footnote{在微信或支付宝上的一网通办小程序、移民局小程序上也可以办理,但 2020 级同学发现小程序上的预约名额很少,很神奇。},注册并登陆后在搜索栏中搜索“普通护照首次申请”或“普通护照换发”。进行普通护照换发的同学需要携带并递交旧护照,旧护照会被当场剪角发还。

需要携带的材料取决于个人情况。若在入学时转到了交大的集体户籍,则只需携带身份证而不需要提供在读证明\footnote{这一点虽然是本人亲眼所见,但还是推荐同学们在开具成绩单时一并开具了在读证明、并尽可能地加盖学校公章,毕竟这只是 2022 年的情况。},上海市公安局会将其视为本市居民,并在七个工作日内办结业务;若没有集体户籍,上海市公安局会将其视为外省市人员,在二十个工作日内办结业务。可以通过提供加盖学校公章(即前文所述的校印)荣升为在沪全日制高等院校在读的外省市人员,上海市公安局同样会在七个工作日内办结业务。

护照申请所需要的中国公民出入境证件申请表和出入境证件电子照可以当场获取,不需要准备。

\subsection{出生公证}

出生公证有两方面作用:一是在办理签证的过程中需要提供,二是抵达法国之后办理 APL(Aide personnalisée au logement
,住房补贴)的时候需要提供双认证的版本,因此是十分重要的文件。办理流程如下: 
\begin{enumerate}
\item 在出生地公证处办理出生公证书,记得要开具中文和法文双语的。所需材料大致如下(可能各地要求有所不同,所有材料均需提供复印件):本人身份证;出生证;户口本(上海交大集体户口的提供户籍证明以及标记有“户籍迁出”的旧户口本);父母身份证;父母结婚证
\item 出生公证书大约一个月内可以制成,建议一次开具两份,方便后面办理其他手续。领取公证书的时候一定要仔细检查里面的个人信息、身份证号码、公证员姓名是否有误,以及仔细对照中文版与法文版有无差异。
\item 拿到出生公证书后,取其中一份拿到出生地的外事部门进行双认证。双认证的意 思是将公证书送至中国外交部和法国驻中国大使馆两个单位进行核实,公证书返回来后法文页的背面会多出外交部与法使馆的签章,这样这本公证书就可以得到法国政府机构的认可了。双认证大约需要
20 个工作日。 
\end{enumerate}
注意:
\begin{enumerate}
\item 按照本人经验,双认证没有必要自己亲自去办(在有的城市也没办法亲自去办),把公证书交给出生地外事部门、市公证处或者承办此类外事工作的公司、旅行社等代为办理是最便捷的方式(具体递交给哪个单位取决于你出生地政府的规定);如果出生地没有外事办的话,那么就去最近的有外事办的地级市办理。
\item 只需要对其中一本公证书进行双认证,另外一本可以留在家中或学校以作备用。
\item 之所以只需要双认证其中一本,是因为目前我们得知的唯一需要双认证的场合是抵达 X 以后办理 APL;另外那本没有双认证的你甚至可以不带到巴黎斯坦来。
\end{enumerate}

\subsection{预签证\label{subsec:=009884=007B7E=008BC1}}

凡赴法长期留学都需要通过 \href{https://www.chine.campusfrance.org/zh-hans}{Campus France}
\footnote{访问这个网站时请选择中文,因为中文版网站上的信息并不仅仅是法语版的翻译,反而多出了很多有用的内容。}的认证,这部分的手续比较繁杂。

注意:Campus France 的网上预签证平台有可能会更新流程,因此下列信息不一定相符,一切问题均以 CF 网站上的解释说明为准。

第一步:注册 \href{https://pastel.diplomatie.gouv.fr/etudesenfrance/dyn/public/authentification/login.html}{Etudes en France}\footnote{下文可能会简称为 EEF。}
帐号,填写个人信息表格\footnote{某特对于 CF 学校列表中没有 EP 表示很惊讶,顿时感到一种野鸡大学之旅的惊慌,如果后续学弟学妹填写信息,法国学校列表中没能找到
X 也请不要恐慌,自己填写学校情况即可}。点击“建立您的 Etudes en France页面”选项,填写相关信息,创建账户,填写相关个人信息。 另外需要填写自己的简历、赴法学习计划以及职业规划,现在可以开始准备起来。 

注意:
\begin{enumerate}
\item 个人地址建议填写交大的地址,否则个人资料会被转至家庭所在地对应的 CF 机构处理,导致不便。
\item 除非有明确指明,个人资料、简历、学习计划、职业规划的填写都需要使用英语或法语。
\item EEF 网站对上传文件的要求很苛刻,需要常备各类压缩、格式转换的网站。例如符合要求的照片应当为 Jpeg 格式,300 DPI,26{*}32
mm,需要裁剪照片以保证尺寸和 DPI 都符合要求,且大小在 35kb 以下。
\end{enumerate}
第二步:参加法语水平考试。两种考试根据日期随机分配,需要在进入 EEF 后进行报名\footnote{因此需要护照发下来以后才有可能预约,而法语考试的预约又要提前一个月......},报名可以到附近的银行柜台携带开户卡身份证,向“法国大使馆的财务处”汇款并备注姓名、CN
号与转账目的\footnote{在登录 EEF 网站后,右上角以 CN- 开头的字符串就是 CN 号。}\footnote{2020 级的同学发现也可以在手机银行上汇款。},随后在自己的
EEF 账号内填写汇款信息,等待自己的汇款被确认。相信学院的法语教学,拿 B1 应该是任何人都可以的!\footnote{考试的时候请听从监考人员指令,否则会被她嫌弃得不得了——改编者记。}考完试后当场出成绩,并会向你发放一张临时成绩单,使用这个成绩单就可以在
EEF 的法语水平上填写了。

备注:正式的成绩单大约需要一个月的时间,领取请携带身份证。(TCF / TEF / DELF / DALF有其一就可以。办理签证一般
B1 即可,但建议一次性考过 B2 ,因为学院的 CTI 证书要求法语达到B2。)

第三步:审核材料和缴费。在所有的个人信息内容全部填写妥当之后,网站上有一处会显示所需填写的全部表格“完整”。这个时候 EEF 会审核材料是否齐全和符合要求,请定期查看
EEF 网站和邮箱以便及时获取消息。如果材料不合要求,EEF 会通知修改。审核材料和缴费是可以同时进行的,以提高办事效率。由于不清楚缴费流程有没有发生变化,在此只提供网上看到的今年新缴费流程的介绍(依旧吁请各位注意,一切以
EEF 以及网上平台的说明解释为准)缴费金额是 1,750 元人民币, 必须通过银行柜台汇款方式进行\footnote{2020 级的同学发现也可以在手机银行上汇款。}。

注意:请携带本人开户身份证,汇款附言要写上自己的姓名 (拼音)、CN 号与转账目的,汇款后需要在 EEF 网站上填写信息, 此处有一个银行代码,填写行号即可。建议大家各自汇款。 

\subsection{签证(上)}

申请签证需要用到两个网站:\href{https://france-visas.gouv.fr/zh/web/france-visas/}{France-Visas}\footnote{下文可能会简称为 FV。}
和 \href{https://fr.tlscontact.com/cn/splash.php}{TLScontact}。它们的作用分别是:FV
网站帮助填写签证申请表,而 TLScontact 网站可以预约递交签证申请的时间。虽然签证最终办结的时间显然应该晚于预签证,但这两个网站的大部分流程并不依赖于预签证。为了节约时间,可以先填写好
FV 的表格但不确认,用得到的 FRA 号注册 TLScontact 网站并预约递交签证申请。\textcolor{red}{请注意,在真正前往签证中心递交材料时,预签证的流程应当完全结束,即收到
EEF 通知预约递交签证申请的站内信,否则将视为材料不全。}

至此,在各种网站上可以进行的操作结束。下面详解签证所需材料:
\begin{description}
\item [{TLScontact的预约单}] 打印出来,在签证中心的柜台出示。
\item [{签证照片}] 任何一个正规的照相馆都可以拍摄;尺寸要求是小二寸、3.5 cm {*} 4.5 cm 白底。
\item [{长期签证申请表原件}] 即 FV 网站的表格,下载后打印出来。如果填写有误,可以在 FV 网站上发起一个新流程并重新填写\footnote{也可以空着到签证中心手写,或者干脆花几百块钱购买高贵的 VIP 服务让签证中心的人修改 / 指导填写并打印。},而并不需要重新预约递交签证申请。
\item [{带有EEF识别码的预注册证明}] 在预签证流程完成后,EEF 网站会提示你将站内信截图,打印出来就是预注册证明。
\item [{OFII申请表(两份)}] 这是一份涉及法国移民局的文件,打印出来按要求填写即可,有一部分是不需要现在填写的。\textcolor{red}{2020
级的学生发现并不需要提交这个表格。}
\item [{出生公证书原件(包括翻译)复印件}] 签证中心只会收走复印件而交还原件。
\item [{法国住宿证明}] 即 Attestation de logement。
\item [{法国学校录取证明}] 即 Attestation d'admission。
\item [{担保书}] 打印出来,由父母签名后复印,注意要有英/法文翻译。
\item [{父母双方的工作证明及复印件}] 由父母单位的人事部门开具中英双语版本即可。需使用公司正式的信头纸并加盖公章、签字、并明确以下信息:任职公司的地址、电话和传真号码;任职公司签字人员的姓名和职务;担保人的姓名、职务、收入、工作年限和工作地点。
\item [{父母的银行存款证明及复印件}] 这个需要回当地的银行开具,跟柜台说是出国留学存款担保用的,就可以开了。如果可以的话,父母双方的都开具一份\footnote{2020 级的同学发现只有一份也没事。},增加材料的可信度(金额不低于
60,000 RMB,冻结至到法国后一个月),签证中心只会收走复印件而交还原件。
\item [{银行对帐单}] 按照要求,信用卡、定期存折和存款证明都是不可以的,所以只能提供三个月以上的借记卡或者活期存折流水;另外,由于你们身份是学生,所以需要提供所打印的流水账单所属人与你的亲属关系证明(出生证明、户口本皆可);同样地,这个也是父母双方的都需要开具,如果无法做到的话需要附上一封解释信(以解释父母某一方材料缺失的原因,如果可能的话附上其他证明材料);\textcolor{red}{流水单开具比较简单,但是一定不要开的太早}\footnote{\textcolor{black}{X16 陈同学由于此问题在递交签证时遇到了困难,不得不回家重新开具,在此,请为这些冲锋陷阵的前辈们致意。}}\textcolor{red}{,尽量银行记录的打印时间在递交签证之内一个月即可。}
\item [{户口本复印件}] 原因如上。
\item [{护照原件复印件}] 十分简单,将所有包含内容(即:假如护照使用过,之前的 visa 面也需要复印)的部分全部复印即可(封面就不用了)。
\item [{航班预定单}] 通常被称为 Electronic Ticket Itinerary。这不是必需的,但是会增加材料的可信度,提高成功率。在收走航班预定单后,最后的材料清单上就会显示:The
applicant has provided an additional supporting document. air ticket
booking
\item [{委托书}] 如果你自己亲自前去递交材料和领取签证,那就不需要提供委托书了。
\item [{保险}] 网站上说要提供保险证明,但 2020 级的同学发现没有这回事。
\end{description}

\subsection{体检与疫苗接种\label{subsec:=004F53=0068C0=004E0E=0075AB=0082D7=0063A5=0079CD}}

如前文所讲,Coralie 会把一系列医疗文件发给你们,主要分为两个部分:体检与疫苗接种。

\subsubsection{体检\protect\footnote{不用办好签证也可以体检,由于疫苗补种,尽量提早体检。但出示签证或法方学校的注册证明会使体检的价格大大降低。}}

首先是一份 Questionnaire médico-biographique,自己填写即可。 

对于 Certificat de non contre-indication à la pratique du sport 和 Certificat
médical confidentiel,需要按照以下方式进行: 

在上海或者家庭所在地(两地选一即可)的出入境检验检疫局的国际旅行卫生保健中心 (或同等资质的单位)做“中国公民赴境外留学体检与预防接种”的体检项目,并把这些医学表格带上。在保健中心做完体检后,将这些体检表留给保健中心,他们会根据你的体检结果进行填写。

如果允许的话,在体检中心索要肺部 X 光照片与纸质报告。

体检完成以后,在一周以后你会拿到一本红色封皮的《国际旅行健康检查证明书》以及当时交过去的表格,请检查表格上是否有医生签名与盖章,并扫描所有的医学表格。记得把这些东西带到法国来。

\subsubsection{疫苗接种}

疫苗接种问题不用太担心,实在来不及接种的疫苗来法国后学校会给免费接种。首先, 让父母速速在家里把你小时候的疫苗接种证找出来(应该是一本十分旧、与出生证同个年代的泛黄的小本子)。

把 Coralie 发送给你的 Certificat de vaccination 打印出来,体检时一起带到保健中心去, 上面列出的疫苗有的是必需疫苗,有的是非必需,需要由医生填写每一种疫苗的初次注射、二次注射与最近一次注射的时间。

X 要求的疫苗种类很多,根据本届经验,将所有的疫苗以及其译名标在这里:
\begin{enumerate}
\item 必需疫苗
\begin{itemize}
\item Poliomyelitis(脊髓灰质炎)
\item Hepatitis A(甲型肝炎)
\item Hepatitis B(乙型肝炎)
\end{itemize}
备注:以上三个,只要你们正常出生在社会主义大家庭,应该都已经打过了,不用重复再打,让医生把日期写上即可。
%
\begin{itemize}
\item Meningitis A + C(流脑苗A+C,上一次注射至少在 5 年内)
\end{itemize}
备注:这个在邮件中的标记是必需疫苗,但是实际需要填的疫苗单里没有这一项……所以被我忽略了,没有造成什么问题。
%
\begin{itemize}
\item DT-Polio (Tetanus Diphtheria Pertussis)(DT-百白破三联),上一次注射至少在10 年内。
\end{itemize}
这个疫苗是个比较大的问题,我们详细说明一下:在中国这个疫苗专供小朋友,成人是不能打这个疫苗的,所以估计你们所有人的上一次注射都在 10
年以前了。针对这一情况, 张学霸友情提供了以下三个选项:(a) 到 X 再接种这个疫苗。(b) 在国内打百白破疫苗加脊髓灰质炎疫苗(糖丸),不推荐此选项因为这个疫苗是给小朋友准备的。\footnote{根据本人经验,每年 12 月 - 2 月期间并非出国高峰,所以整个大粤国也只有广州和深圳“有可能”有百白破疫苗提供(保健中心医生语),因此这一选项的可行性也很低。——作者注。}(c)
妥善利用疫苗接种证明开具过程的漏洞,修(wei)改(zao)此疫苗的接种时间。
%
\item 非必需疫苗
\begin{itemize}
\item RUBEOLLE(Fr,风疹)
\item ROUGEOLE(Fr,麻疹)
\item OREILLONS(Fr,腮腺苗)
\item BCG(Fr,Vaccin biliéde Calmette et Guérin ,卡介苗)
\end{itemize}
备注: 以上四个在小时候应该也打过了,而且没有时间限制。
%
\begin{itemize}
\item IDR Tuberculine 5TU(Fr,中等强度结核菌素(5TU)(标准试验剂量)
\end{itemize}
备注: 这个实际是一次皮试,我小的时候没有做过所以补做了一次,从皮试到复查完成前后大约花了 7 天。\footnote{根据 X2016 同学的经验,此过程由于小时候接种情况各有差异,最长时间可能需要 14 天到 21 天不等,因此建议提前去体检。}
\end{enumerate}
注意:在接种任何疫苗之前,记得咨询好医生,留出足够的时间补种缺漏的疫苗,以免出现疫苗之间相互影响的问题。

确定所需的疫苗都已经注射完毕后,保健中心会把你的疫苗接种记录转录到一本黄色封皮的《疫苗接种或预防措施国际证书》上。让保健中心的医生填写好各个疫苗的接种时间,
并且签名盖章。这两样东西记得带到法国来。

\section{目前还无能为力、将来会发生的事情}

\subsection{签证(中)}

CF 预签证审批成功后,就可以在指定的日期和时间\footnote{提前到达会被拦在门外,直到预约的时间才可以进去。}到外企德科上海办公室递交签证申请材料。递交材料的过程十分轻松愉快,签证官不会对你进行面试,偶尔会跟你闲聊几句,你认真回答他
/ 她就好。签证官会当场检查所有材料,并核对原件复印件是否相符,拿走他需要的所有文件,剩下的材料(例如银行对帐单、公证书、录取通知书等)会当场还给你。护照此时会被他们收走。

之后去柜台交签证费(可以刷卡,方便快捷),柜台会给你一张届时回来取护照的回执, 请务必保存好。付款完记得拿上发票。继续向前,你将需要领号等待录取生物信息,拍照\footnote{此照片即为绿油油的 visa 照片。}录取指纹后你就可以轻松愉快地离开大楼返回学校了(顺路吃个饭也不错\footnote{在大楼和地铁站之间有一个商业中心,里面有不少可以吃饭的地方。——作者注。})。 

\subsection{签证(下)}

按照外企德科网站的说法(以及我们一点人生的经验),大约(根据不同人员的颜值) 10 个工作日以内外企德科会在网站上通知你去签证中心领回你的护照(现已全面支持
EMS 全国邮寄,这会使得拿到签证时间增加两天左右)。记得带上回执,拿到的是一个信封,你可以当场打开,一张绿油油印有你大脸签证照片的法国本土
Visa 就会刺瞎你们的狗眼。请当场核对一下姓名、护照号、生效日期与有效期(一年)是否有误。 

没有问题的话,签证一事就此完成,生活再也没有压力。 

备注:我们拿到的是一年期的 D 类学生签证,没有标记申根国均可通行。抵达 X 以后, Coralie 会安排你们到 OFII(Office
Français de l'Immigration et de l'Int égration)面试,当场贴上一张 vignette
之后就具有了申根区通行权,有效期一年,失效前三个月 Coralie 会通知你们去办 Titre de Séjour (长居身份证),并在今后取代护照的作用。

\subsection{交大手续(下)}

终于可以退学了,在离校前需要在离校单上盖好所有章(社会主义官僚国家事情就是这么多,当然法国也是一个官僚国家,恭喜你们脱离苦海进入新苦海),主要包括:
\begin{itemize}
\item 共党分子们,去新行政楼 A 楼 2 楼的校党委组织部盖“党籍保留”章;共青团分子们先去组织部问一下,应该在学院楼的学院团委即可盖章;
\item 去学生服务中心的宿管部盖章
\item 去户籍科盖章;
\item 去图书馆还清借书并盖章;
\item 学生服务中心二楼学生事务中心盖章\footnote{只需要检查你是否有享受国家补助(否则是不能出国的),学生事务中心的工作人员在翻箱倒柜一通之后终于找出了相应的章并且盖了上去。——作者注。};
\item 新行政楼 B 楼一楼财务处盖章
\item 校园卡服务中心销卡并盖章\footnote{如果你是怀旧的同学,建议提前补办校卡,否则注销学生卡的时候存在收回校园卡的可能性。}\footnote{2020 级同学的校园卡并没有被回收。}。
\end{itemize}
集齐以上 7 个章后把离校单交给学院负责派出的老师,本次大型魔幻现实主义之中国特色社会主义盖章之旅就结束了。

\section{结语}

至此,你们需要办的所有手续全部完成,Facebook 可以注册起来了。肉翻即将成功, 各位仍需努力。一个人的命运,当然要靠自我奋斗,但是也要考虑历史的行程。我与这篇文章的作者之间没有任何关系,将来宣传上要是有偏差,你们要负责的。

\section{出国各项费用单(仅供参考与娱乐)}

\begin{tabular}{|c|c|}
\hline 
项目名称 & 金额(RMB)\tabularnewline
\hline 
\hline 
TCF / e - TEF 考试费用 & 1,890\tabularnewline
\hline 
Campus France 审批 & 2,100\tabularnewline
\hline 
成绩认证 & 300\tabularnewline
\hline 
签证费用(+ 快递非上海) & 672(+90) \tabularnewline
\hline 
出生公证 & 160\tabularnewline
\hline 
出生公证双认证 & 约 300\tabularnewline
\hline 
体检 + 疫苗 & 约 1,000\tabularnewline
\hline 
中英文成绩单 & 5 / 份 \tabularnewline
\hline 
单程机票 & 约 5,000\tabularnewline
\hline 
\end{tabular}

合计:约 10,000RMB
\end{document}
